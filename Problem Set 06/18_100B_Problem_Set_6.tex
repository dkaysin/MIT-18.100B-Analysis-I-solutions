\documentclass{article}
\usepackage[utf8]{inputenc}
\usepackage[english]{babel}
\usepackage[]{amsthm}
\usepackage[]{amssymb} 
\usepackage{amsmath}
\usepackage{tcolorbox}
\usepackage{mathtools}

\newcommand{\R}{\mathbb{R}}
\newcommand{\N}{\mathbb{N}}
\DeclarePairedDelimiter{\norm}{\lVert}{\rVert}

\theoremstyle{remark}
\newtheorem{claim}{Claim}


\title{18.100B: Problem Set 6}
\author{Dmitry Kaysin}
\date{January 2020}
\begin{document}
\maketitle 


\subsection*{Problem 1}

\begin{tcolorbox}
Prove that if $\sum |a_n|$ converges, then $\sum |a_n|^2$ also converges.
\end{tcolorbox}

\begin{proof}

Since  $\sum |a_n|$ converges $\lim_{n \to \infty} |a_n|$ must be equal to $0$, which implies that for any $\epsilon>0$ there exists $N \in \N$ such that $|a_n| - 0 < \epsilon$ for all $n > N$.
Consider such $N$ for $\epsilon = 1$.

Since series $\sum |a_n|$ converges, its subseries $\sum_{n=N+1}^{\infty}|a_n|$ must also converge.
We notice that $|a_n|^2 < |a_n|$ for all $n>N$ since $|a_n| < 1$. Therefore $\sum_{n=N+1}^{\infty}|a_n|^2$ must also converge.

We then notice that series $\sum |a_n|^2$ can be written as a sum of two expressions: 
$$ \sum_{n=1}^{\infty} |a_n|^2 = \sum_{n=1}^{N} |a_n|^2 + \sum_{n=N+1}^{\infty} |a_n|^2 $$.

The second expression is a convergent series, as we have just proved. The first expression must be finite since $N$ is finite. We conclude that $\sum |a_n|^2$ converges.

\end{proof}


\subsection*{Problem 2}

\begin{tcolorbox}
Prove that 
$$ \sum_{n=1}^{\infty} \frac{1}{n(n+1)(n+2)} = \frac{1}{4} $$
\end{tcolorbox}

\begin{proof}

We can write 
\[ \frac{1}{n(n+1)(n+2)} = \frac{1}{2n(n+1)}-\frac{1}{2(n+1)(n+2)} \]
Denoting 
\[ b_n = \frac{1}{2n(n+1)} \]
we can see that
\[ \sum_{n=1}^{\infty} \frac{1}{n(n+1)(n+2)} =  \sum_{n=1}^{\infty} b_n - b_{n+1} = b_1 = \frac{1}{2 \cdot 1 \cdot 2} = \frac{1}{4} \]

\end{proof}


\subsection*{Problem 3}

\begin{tcolorbox}
Investigate the behavior (convergence and divergence) of $\sum_{n=1}^\infty a_n$ if

a) $ a_n = \sqrt{n+1} - \sqrt{n}$
\end{tcolorbox}

Answer: Series $\sum a_n$ diverges.

\begin{proof}

Using telescoping argument, we can rewrite 
\[ \sum_{n=1}^k a_n = \sqrt{1} + \sqrt{k+1} \]
which clearly diverges as $k \to \infty$.

\end{proof}

\begin{tcolorbox}
b) $ a_n = \frac{\sqrt{n+1} - \sqrt{n}}{n}$
\end{tcolorbox}

Answer: Series $\sum a_n$ converges.

\begin{proof}

\[ a_n = \frac{\sqrt{n+1} - \sqrt{n}}{n} = \frac{n+1-n}{n(\sqrt{n+1}+\sqrt{n})} < 
\frac{1}{n \sqrt{n}} = \frac{1}{n^{\frac{3}{2}}} \]

Series $\sum \frac{1}{n^{\frac{3}{2}}} $ is convergent by Rudin 3.28, therefore series $\sum a_n$ is also convergent by comparison test.

\end{proof}

\begin{tcolorbox}
c) $a_n = \frac{1}{1+\alpha^n}$ where $\alpha \geq 0$ is some fixed number
\end{tcolorbox}

Answer: Series $\sum a_n$ diverges for $0 \leq \alpha \leq 1$ and converges for $\alpha > 1$.

\begin{proof}

If $\alpha = 0$ then $a_n = 1$, therefore $\sum a_n$ diverges.
If $0 < \alpha < 1$ then $\alpha^n \to 0$ and $a_n \to 1$, therefore $\sum a_n$ diverges.
If $\alpha = 1$ then $\alpha^n = 1$ and $a_n = \frac{1}{2}$, therefore $\sum a_n$ diverges.
If $\alpha > 1$ then $\alpha^n \to \infty$ and $a_n \to 0$.
Furthermore
\[ \frac{1}{1+\alpha^n} < \frac{1}{\alpha^n} = \left( \frac{1}{\alpha} \right)^n \]
We notice that $\sum (\frac{1}{\alpha})^n$ is a convergent geometric series for $\alpha>1$ by Rudin 3.26.
In this case $a_n$ is also convergent by comparison test.

\end{proof}


\subsection*{Problem 4}

\begin{tcolorbox}
Show that convergence of $\sum_{n=1}^\infty a_n$ with $a_n \geq 0$ implies convergence of 
\[ \sum_{n=1}^\infty \frac{\sqrt{a_n}}{n} \]
\end{tcolorbox}

\begin{proof}

Following Cauchy–Schwarz inequality we notice that
\[ \left( \sum \sqrt{a_n} \cdot \frac{1}{n} \right)^2 \leq \sum a_n \cdot \sum \frac{1}{n^2} \]
Since both $\sum a_n$ and $\sum \frac{1}{n^2}$ converge, $\sum_{n=1}^\infty a_n \cdot \sum_{n=1}^\infty \frac{1}{n^2}$ must be a finite number.
Therefore, $\left( \sum_{n=1}^\infty \sqrt{a_n} \cdot \frac{1}{n} \right) ^ 2$ is also finite.
We conclude that the series $\sum \frac{\sqrt{a_n}}{n}$ must converge.
\end{proof}


\subsection*{Problem 5}

\begin{tcolorbox}
Assume $a_0 \geq a_1 \geq a_2 \geq \dots$ and suppose that $\sum a_n$ converges. Prove that
$$ \lim_{n \to \infty} (n a_n) = 0 $$
\end{tcolorbox}

\begin{proof}

We first notice that $a_n$ is monotonically non-increasing.
Since $\sum a_n$ converges, $a_n \to 0$ as $n \to \infty$.
Therefore all $a_n$ are non-negative.

We then notice that $a_{k} \leq a_n$ for any $n < k$. Therefore
\[ n a_{2n} \leq \sum_{k=n+1}^{2n} a_k \]
and 
\[ n a_{2n-1} \leq \sum_{k=n}^{2n-1} a_k \]

Since $\sum a_n$ converges, for arbitrary $\epsilon>0$ we can always find such integer $N$ that $\sum_{k=N+1}^{2N} a_k < \frac{\epsilon}{2}$ (by Rudin 3.22).
We find that
\begin{gather*}
    N a_{(2N)} \leq \sum_{k=N+1}^{2N} a_k < \frac{\epsilon}{2} \\ 
    (2N) a_{(2N)} < \epsilon    
\end{gather*}
This is equivalent to subsequence $(2n) a_{(2n)}$ converging to a limit of $0$.

Furthermore, for arbitrary $\epsilon>0$ we can always find such integer $N$ that $\sum_{k=N}^{2N-1} a_k < \frac{\epsilon}{2}$.
We find that
\begin{gather*}
    N a_{(2N-1)} \leq \sum_{k=N}^{2N-1} a_k < \frac{\epsilon}{2} \\
    (2N-1) a_{(2N-1)} < (2N) a_{(2N-1)} < \epsilon     
\end{gather*}
This is equivalent to subsequence $(2n-1) a_{(2n-1)}$ converging to a limit of $0$.

Since both even and odd subsequences of sequence $(n a_n)$ converge to $0$, $a_n$ itself must converge to $0$. 

\end{proof}


\subsection*{Problem 6}

\begin{tcolorbox}
If $X$ and $Y$ are metric spaces and $f : X \to Y$ is a mapping between them, show that the following statements are equivalent:

a) $f^{-1} (B)$ is open in $X$ whenever $B$ is open in $Y$.

b) $f^{-1} (B)$ is closed in $X$ whenever $B$ is closed in $Y$.

c) $f(\overline{A}) \subseteq \overline{f(A)}$ for every subset $A$ of $X$.
\end{tcolorbox}

\begin{claim}
For any set $B \subseteq Y$ and map $f$:
\[ f^{-1}(B^c) = (f^{-1}(B))^c \]
\end{claim}

\begin{proof}

Consider arbitrary $x \in f^{-1}(B^c)$:
\begin{gather*}
    f(x) \in B^c \\
    f(x) \notin B \\
    x \notin f^{-1}(B) \\
    x \in (f^{-1}(B))^c     
\end{gather*}
Since all above steps are bidirectional ("if and only if"), we conclude that
\[ f^{-1}(B^c) = (f^{-1}(B))^c \]

\end{proof}

We then prove equivalence of properties a) and b).

\begin{proof}

Suppose a) holds.
Consider a closed set $B \subseteq Y$.
Its complement in $Y$ must be open.
Pre-image of $B^c$ must also be open by property a).
Its complement, $(f^{-1}(B^c))^c$, is closed in $X$.
Using Claim 1 we find that $(f^{-1}(B^c))^c = f^{-1}(B)$. Therefore $f^{-1}(B)$ must be closed.
Proof in the other direction follows the same argument (via considering open set $B$ instead of closed set $B$).

\end{proof}

\begin{claim}
For arbitrary $B \in Y$ and map $f$:
\[ f(f^{-1}(B)) \subseteq B \]
\end{claim}

\begin{proof}

Consider arbitrary $y \in f(f^{-1}(B))$.
Choose $x \in f^{-1}(B)$ such that $f(x) = y$.
\begin{gather*}
    x \in f^{-1}(B) \\
    f(x) \in B \\
    y \in B    
\end{gather*}

\end{proof}

\begin{claim}
For arbitrary $A \in X$ and map $f$:
\[ A \subseteq f^{-1}(f(A)) \]
\end{claim}

\begin{proof}

Consider arbitrary $x \in A$:
\begin{gather*}
    f(x) \in f(A) \\
    x \in f^{-1}(f(A))    
\end{gather*}

\end{proof}

We will now prove equivalence of properties b) and c).

\begin{proof}

Suppose b) holds. Consider arbitrary subset $A \subseteq X$. 
\[ A \subseteq f^{-1}(f(A)) \]
Furthermore we notice that
\[ A \subseteq f^{-1}(f(A)) \subseteq f^{-1}(\overline{f(A)}) \]
Since $\overline{f(A)}$ is closed, its pre-image $f^{-1}(\overline{f(A)})$ must also be closed by property b).
Closure of $A$ is the smallest closed set containing $A$, in other words, any closed set that contains $A$ contains $\overline{A}$.
Therefore:
\begin{gather*}
    \overline{A} \subseteq f^{-1}(\overline{f(A)}) \\
    f(\overline{A}) \subseteq f(f^{-1}(\overline{f(A)}))
\end{gather*}

Using Claim 2 we conclude that
\[ f(\overline{A}) \subseteq \overline{f(A)} \]

Now suppose c) holds.
For arbitrary closed subset $B \subseteq Y$ consider $f^{-1}(B) \subseteq X$. Considering property c):
\[ f(\overline{f^{-1}(B)}) \subseteq \overline{f(f^{-1}(B))} \]
Using Claim 2:
\[ f(\overline{f^{-1}(B)}) \subseteq \overline{B} \]
Since $B$ is closed $B = \overline{B}$, thus:
\begin{gather*}
    f(\overline{f^{-1}(B)}) \subseteq B
    f^{-1}(f(\overline{f^{-1}(B)})) \subseteq f^{-1}(B)
\end{gather*}
Using Claim 3:
\[ \overline{f^{-1}(B)} \subseteq f^{-1}(B) \]
We conclude that $f^{-1}(B)$ is closed since it contains its closure.

\end{proof}


\subsection*{Problem 7}

\begin{tcolorbox}
Suppose that $f : \R \to \R$ satisfies
\[ \lim_{h \to 0} (f(x+h)-f(x-h)) = 0 \]
for every $x \in \R$. Does this imply that $f$ is continuous?
\end{tcolorbox}

Answer: No.

\begin{proof}

Counterexample: consider function
\[ f(x) = \frac{1}{x^2}. \]
It is clearly discontinuous at $x=0$.
However, it satisfies the stated condition.
Specifically, if we evaluate $\lim_{h \to 0} f(x+h)-f(x-h)$ at $x=0$ we find
\[ \lim_{h \to 0} \frac{1}{(0+h)^2} - \frac{1}{(0-h)^2} = 0 \]
as required.

\end{proof}


\end{document}
