\documentclass{article}
\usepackage[utf8]{inputenc}
\usepackage[english]{babel}
\usepackage[]{amsthm}
\usepackage[]{amssymb} 
\usepackage{amsmath}
\usepackage{tcolorbox}

\newtheorem{claim}{Claim}


\title{18.100B: Problem Set 7}
\author{Dmitry Kaysin}
\date\today

\begin{document}
\maketitle 

\subsection*{Problem 1}

\begin{tcolorbox}
Consider an infinite series with alternating signs
$$ \sum_{n=0}^\infty (-1)^n a_n $$
Prove the Leibnitz criteria for convergence:

If $a_i > 0$ for all $i$, $a_i+1 \leq a_i$ and $\lim_{i \to \infty} a_i = 0$, then the series converges.
\end{tcolorbox}

Define for $n \in \mathbb{N} \cup \{0\}$:
$$ s_n = \sum_{k=0}^{n} (-1)^k a_k $$

Claim 1: Odd subsequence of $s_n$ is monotonically increasing.
\begin{proof}
We notice that for any $n$
$$ s_{2n+3} = s_{2n+1} + a_{2n+2} - a_{2n+3} $$
Since $a_{2n+3} \leq a_{2n+2}$ we conclude that for any $n$:
$$ s_{2n+1} \leq s_{2n+3} $$
\end{proof}

Claim 2: Even subsequence of $s_n$ is monotonically decreasing.
\begin{proof}
$$ s_{2n+2} = s_{2n} - a_{2n+1} + a_{2n+2} $$
Since $a_{2n+2} \leq a_{2n+1}$ we conclude that for any $n$:
$$ s_{2n+2} \leq s_{2n} $$
\end{proof}

Claim 3: For any $n \in \mathbb{N} : s_{2n-1} < s_{2n}$.
\begin{proof}
$$ s_{2n} - s_{2n-1} = s_{2n-1} + (-1)^{2n} a_{2n} - s_{2n-1} = a_{2n} > 0 $$
\end{proof}

Claim 4: Any element of the odd subsequence is not greater than any element of the even subsequence.
\begin{proof}
Consider $n, k \in \mathbb{N}$. Assume, without loss of generality, $n > k$.
Using Claims 1, 2 and 3 we can see that:
$$ s_{2k-1} < s_{2n-1} < s_{2n} < s_{2k} $$
We conclude that
$$ s_{2k-1} < s_{2n} $$
and
$$ s_{2n-1} < s_{2k} $$
\end{proof}

Conclusion: Both odd and even subsequences of $s_n$ converge to the same limit, thus sequence $s_n$ converges.
\begin{proof}
Odd subsequence of $s_n$ is monotonically increasing and is bounded by any element of the even subsequence, therefore it converges. Using the same argument, even subsequence also converges.
We also notice that difference between corresponding elements of odd and even sequence becomes arbitrarily small since $s_{2n} - s_{2n-1} = a_{2n}$ and $a_{2n} \to 0$. Therefore, odd and even subsequences of $x_n$ must converge to the same limit.
\end{proof}


\subsection*{Problem 2}

\begin{tcolorbox}
We say that a rational number $r > 0$ is written in reduced form $r = \frac{p}{q}$ if $p$ and $q$ are positive integers with no common factor. Consider the function defined on $(0, 1)$ by
$$ f(x) =
\begin{cases}
    0, & \text{if $x$ is irrational}\\
    \frac{1}{q}, & \text{if $x=\frac{p}{q}$ in reduced form}
\end{cases}
$$
Prove that $f$ is continuous at every irrational $x$, but discontinuous at every rational $x$.
\end{tcolorbox}

\begin{proof}
Consider an arbitrary irrational $x$ and an arbitrary $\epsilon>0$. We will show that there exists $\delta$-neighbourhood of $x$ such that image of all points from such neighbourhood is a subset of $B_{\epsilon}(f(x)) = B_{\epsilon}(0)$.

Choose any $n \in \mathbb{N}$ such that $\frac{1}{n} < \epsilon$. For this $n$ we can always find $k \in \mathbb{N}$ such that interval
$$ E = \left( \frac{k-1}{n}, \frac{k+1}{n} \right) $$
contains $x$. We notice that for any rational number $\frac{p}{q} \in E$: $q > n $. Therefore:
$$ f \left( \frac{p}{q} \right) = \frac{1}{q} < \frac{1}{n} < \epsilon $$.

Consider any $\delta$ such that $B_{\delta}(x)$ is a subset of $E$. Points of $B_{\delta}(x)$ are either rational or irrational. Image of all rational points of $B_{\delta}(x)$ is a subset of $B_{\epsilon}(0)$ as proven above. Image of all irrational points in any neighbourhood of $x$ is $\{0\}$, which is definitely a subset of $B_{\epsilon}(0)$. Therefore, $f$ must be continuous at $x$. We conclude that $f$ is continuous at all irrational $x \in \mathbb{R}$.

Consider an arbitrary rational $x$. We notice that $f(x)>0$. Any $\delta$-neighbourhood of $x$ contains irrationals. Image of such irrationals under $f$ is $\{0\}$, which lies outside of any $\epsilon$-neighbourhood of $f(x)$ for all $0 < \epsilon < f(x)$. Therefore $f$ cannot be continuous at rational $x \in \mathbb{R}$.

\end{proof}


\subsection*{Problem 3}

\begin{tcolorbox}
Let $f$ and $g$ be continuous mappings of a metric space $\mathcal{M}$ into a metric space $\mathcal{N}$, and let $Q \subseteq \mathcal{M}$ be a dense subset of $\mathcal{M}$.

a) Prove that $f(Q)$ is dense in $f(\mathcal{M})$.
\end{tcolorbox}

We will prove that every point of $f(\mathcal{M})$ is either an element of $f(Q)$ or a limit point of $f(Q)$.

\begin{proof}

Consider arbitrary $x \in \mathcal{M}$. Since $Q$ is dense in $\mathcal{M}$, either $x$ is an element of $Q$ or a limit point of $Q$. In the former case, $f(x) \in f(Q)$. Now consider the latter case.

Since $f$ is a continuous mapping, for any $B_{\epsilon}(f(x)) \subseteq f(\mathcal{M})$ we can always find $B_{\delta}(x) \in \mathcal{M}$ such that $f (B_{\delta}(x)) \subseteq B_{\epsilon}(f(x))$.

Since $x$ is a limit point of $Q$, $B_{\delta}(x)$ must contain at least one point of $Q$ other than $x$, denote it $q$. We notice that $f(q) \in B_{\epsilon}(f(x))$. If $f(x) = f(q)$ then $f(q) \in f(Q) \iff f(x) \in f(Q)$. If $f(x) \neq f(q)$ then $B_{\epsilon}(f(x))$, which we have chosen arbitrarily, contains a point of $f(Q)$ other than $f(x)$, namely $f(q)$. We conclude that image of every point of $\mathcal{M}$ is either an element of $f(Q)$ or a limit point of $f(Q)$. Therefore, $f(Q)$ is dense in $f(\mathcal{M})$. 

\end{proof}

\begin{tcolorbox}
b) Show that if $f (x) = g (x)$ for every $x \in Q$, then $f (x) = g (x)$ for every $x \in \mathcal{M}$.
\end{tcolorbox}

\begin{proof}
Suppose there exists some $x \in \mathcal{M}$ outside $Q$ such that $f(x) \neq g(x)$. Since $\mathcal{N}$ is metric and thus Hausdorff, we can find two disjoint open balls around points $f(x)$ and $g(x)$, namely $E_f = B_{\epsilon}(f(x))$ and $E_g = B_{\epsilon}(g(x))$. Since both $f$ and $g$ are continuous, we can find open ball $B_{\delta}(x) \subseteq \mathcal{M}$ such that $f (B_{\delta}(x)) \subseteq E_f$ and $g (B_{\delta}(x)) \subseteq E_g$.

Since $Q$ is dense in $\mathcal{M}$, $B_{\delta}(x)$ must contain at least one point of $Q$; denote it $q$. Under the initial assumption, $f(q)=g(q)$. However, this cannot be the case since $E_f$ and $E_g$ are disjoint. Contradiction. We conclude that no such points $x$ exist.

\end{proof}


\subsection*{Problem 4}

\begin{tcolorbox}
Suppose that $f : E \to \mathbb{R}$ is a continuous map from some subset $E \subseteq \mathbb{R}$.

a) Show that it is possible to have $E$ bounded and $f(E)$ unbounded.
\end{tcolorbox}
\begin{proof}
Example: $f(x) = \tan x$ for $x \in \left( -\dfrac{\pi}{2}, \dfrac{\pi}{2} \right)$.
\end{proof}

\begin{tcolorbox}
b) Show that if $f$ is uniformly continuous and $E$ is bounded, then $f(E)$ must be bounded.
\end{tcolorbox}
\begin{proof}
Consider arbitrary $\epsilon>0$. Since $f$ is uniformly continuous we can find $\delta>0$ such that images of any two points that are within $\delta$ distance of each other are within $\epsilon$ distance of each other.
Since $E$ is bounded, we can find its finite cover $\mathcal{U}$ such that for every two points $x,y$ in every set of such cover $d(x,y) < \delta$. Since distance between images of these two points $d(f(x), f(y))$ is smaller than $\epsilon$, image of each of the set in collection $\mathcal{U}$ is bounded. Union of images of such sets covers $f(E)$ and is bounded since there is only finitely many of such images. Therefore, $f(E)$ must be bounded.
\end{proof}

\begin{tcolorbox}
c) Show that it is possible to have $f$ uniformly continuous, $E$ unbounded, and $f(E)$ unbounded.
\end{tcolorbox}
\begin{proof}
Example: $f(x) = x$.

This function is indeed uniformly continuous since
$$ d(x,y) = |x-y| $$
$$ d(f(x),f(y)) = |x-y| $$
and therefore for any $\epsilon > 0$ we can select $\delta < \epsilon$ such that for any $x, y$ with $d(x,y)<\delta$ we have that $d(f(x),f(y)) < \epsilon$.


\end{proof}

\subsection*{Problem 5}

\begin{tcolorbox}
Suppose that $f : \mathcal{M} \to \mathcal{N}$ is a uniformly continuous mapping between metric spaces.

a) Prove that if $(x_n)$ is a Cauchy sequence in $\mathcal{M}$, then $(f (x_n))$ is a Cauchy sequence in $\mathcal{N}$.
\end{tcolorbox}
\begin{proof}
Since $f$ is uniformly continuous, for arbitrary $\epsilon>0$ there exists $\delta>0$ such that for any $a,b \in \mathcal{M}$ with $d(a,b)<\delta$ we have $d(f(a),f(b))<\epsilon$. Since $x_n$ is Cauchy sequence, there exists $N\in\mathbb{N}$ such that $d(x_k, x_m) < \delta$ for all $k>N, \>m>N$. Therefore, $d(f(x_k), (x_m)) < \epsilon$. We conclude that $f(x_n)$ is Cauchy sequence.
\end{proof}

\begin{tcolorbox}
b) Use the function $g : \mathbb{R} \to \mathbb{R}$, $g(x) = x^2$ to show that it is possible for a continuous function to send Cauchy sequences to Cauchy sequences without being uniformly continuous.
\end{tcolorbox}
First, we notice that $g(x)=x^2$ is continuous but not uniformly continuous. 
\begin{proof}
Indeed, for any $\delta>0$ we can find $x$ such that $d(g(x+\delta),g(x))$ is greater than any $\epsilon > 0$.
$$ d(g(x+\delta),g(x)) = (x+\delta)^2 - x^2 = 2x\delta + \delta^2 $$
$$ 2x\delta + \delta^2 > \epsilon $$
$$ x > \frac{\epsilon -\delta^2}{2\delta} $$
\end{proof}
If $x_n$ is Cauchy sequence then $g(x_n)=x_n^2$ is also Cauchy sequence.
\begin{proof}

Consider arbitrary $\epsilon > 0$. Since $x_n$ is Cauchy sequence, it must be bounded and have supremum $s = \sup x_n$. Furthermore, there must exist $N \in \mathbb{N}$ such that:
$$ d(x_k, x_m) < \frac{\epsilon}{2s} \text{ for any } k > N, \> m > N $$

Consider distance between corresponding elements of $g(x_n)$. Suppose, without loss of generality, that $x_k > x_m$, then
$$ d (g(x_k), g(x_m) = x_k^2 - x_m^2 = (x_k - x_m) (x_k + x_m) <  \frac{\epsilon}{2s} (x_k+x_m) $$
Notice that $x_k \leq s$ and $s_m \leq s$ since $s$ is supremum of $x_n$. Therefore
$$ d (g(x_k), g(x_m)) < \epsilon $$
for all $k > N, m > N$. We conclude that $g(x_n)$ is Cauchy sequence.

\end{proof}


\subsection*{Problem 6}

\begin{tcolorbox}
Let $E$ be a non-empty subset of a metric space $\mathcal{M}$, define the distance from $x \in X$ to $E$ by
$$ d_E(x) = \inf_{z \in E} d(x,z) $$
Prove that $d_E$ is a uniformly continuous function on $X$, by showing that
$$ | d_E(x) - d_E(y) | \leq d(x,y) $$
\end{tcolorbox}
\begin{proof}
Consider arbitrary points $x,y \in X$ and arbitrary point $e \in E$. Following triangle inequality:
$$ d(x,e) \leq d(x,y) + d(y,e) $$
Following the definition, $d_E(x) \leq d(x,e)$, thus 
$$ d_E(x) \leq d(x,y) + d(y,e) $$
$$ d_E(x) - d(x,y) \leq d(y,e) $$
This holds for any point $e$, thus for $\inf_{e \in E}(y,e) = d_E(y)$:
$$ d_E(x) - d(x,y) \leq d_E(y) $$
$$ d_E(x) - d(x,y) \leq d_E(y) $$
$$ d_E(x) - d_E(y) \leq d(x,y) $$
Switching $x$ and $y$ we find that
$$ d_E(y) - d_E(x) \leq d(x,y) $$
Thus
$$ | d_E(x) - d_E(y) | \leq d(x,y) $$
Now consider arbitrary $\epsilon > 0$. We can always choose $\delta = \epsilon$ such that for any $x,y \in X$ with $d(x,y)<\delta$ we have
$$ d(d_E(x), d_E(y)) = |d_E(x)-d_E(y)| \leq d(x,y) < \delta = \epsilon $$
Thus $d_E(x)$ is uniformly continuous.

\end{proof}


\subsection*{Problem 7}

\begin{tcolorbox}
Suppose $K$ and $F$ are disjoint subsets of $\mathcal{M}$, such that $F$ is closed and $K$ is compact. Prove that there exists a $\delta > 0$ such that $d (p, q) > \delta$ whenever $p \in F$ and $q \in K$.
\end{tcolorbox}
\begin{proof}
Consider function $d_F=\inf_{z \in K} d(x,z)$ defined on $K$. This is a uniformly continuous and, thus, continuous function (see Problem 6).

Image of compact set $K$ under continuous function $d_F$ is also compact (Rudin 4.14). Therefore, we can find minimum for $d_F(K) \subseteq \mathbb{R}$. We notice that $\min d_F(K) > 0$ because otherwise we would have at least one point $q \in K$ such that $d_F(q) = 0$ and, therefore, $q \in F$ since $F$ is closed, which cannot be the case since $F$ and $K$ are disjoint. Therefore, for any $p \in F, q \in K$:
$$ d(p,q) \geq d_F(q) \geq \min d_F(K) > 0 $$
or 
$$ d(p,q) > \delta$$
for some $\delta > 0$.
\end{proof}

\begin{tcolorbox}
 Show that the conclusion can fail if we only assumed that $K$ was closed, instead of compact.
\end{tcolorbox}
\begin{proof}
Counterexample: $F = \left\{ \left(x,\frac{1}{x}\right) : x\in\mathbb{R} \right\}$, $K = \left\{ \left(x,-\frac{1}{x}\right) : x\in\mathbb{R} \right\}$. Both $F$ and $K$ are closed in $\mathbb{R}^2$ and unbounded, thus not compact. However, for any $\delta$ we can find points $\left( z, \frac{1}{z} \right)$, $\left( z, -\frac{1}{z} \right)$ such that distance between them is less than $\delta$:
$$ \frac{1}{x} - \left( - \frac{1}{x} \right) = \frac{2}{x} < \delta $$
$$ x > \frac{2}{\delta}. $$
\end{proof}

\end{document}
