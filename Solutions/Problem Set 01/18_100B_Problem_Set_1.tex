\documentclass{article}
\usepackage[utf8]{inputenc}
\usepackage[english]{babel}
\usepackage[]{amsthm} 
\usepackage[]{amssymb} 
\usepackage{tcolorbox}

\title{18.100B: Problem Set 1}
\author{Dmitry Kaysin}
\date\today

\begin{document}
\maketitle 

\subsection*{Problem 1}

\begin{tcolorbox}
Prove that there is no rational number whose square is 12.
\end{tcolorbox}

\begin{proof}
Suppose there exist $p,q \in \mathbb{Z}$ such that $p$ and $q$ are relatively prime and $(\frac{p}{q})^2=12$.
$$p^2 = (4)(3)q^2$$
This means that 3 divides $p^2$, which is only possible if 3 divides $p$, or $p = 3k$.
$$p^2 = 9k^2 = (4)(3)q^2$$
$$3k^2=4q^2$$
This means that 3 divides $q^2$ and 3 divides $q$.
We've arrived at conclusion that 3 divides both $p$ and $q$, a contradiction to the assumption that $p$ and $q$ are relatively prime. Thus, there is no rational number whose square is 12.
\end{proof}

\subsection*{Problem 2}
\begin{tcolorbox}
Let $S$ be a non-empty subset of the real numbers, bounded above. Show that if $u = \sup S$, then for every natural number $n$, the number $u - \frac{1}{n}$ is not an upper bound of $S$,
but the number $u + \frac{1}{n}$ is an upper bound of $S$.
\end{tcolorbox}

\begin{proof}
If $u$ is the least upper bound of $S$ then no upper bound of $S$ is larger than $u$. Suppose $u-\frac{1}{n}$ is an upper bound of $S$. But $u-\frac{1}{n} < u$. Then $u$ cannot be the least upper bound. Contradiction.

Since $u$ is the least upper bound of $S$ then $s \leq u$ for all $s \in S$. Therefore $s \leq u < u+\frac{1}{n}$ for any $n \in \mathbb{N}$ and $u+\frac{1}{n}$ is an upper bound of $S$ by definition.
\end{proof}

\subsection*{Problem 3}

\begin{tcolorbox}
Show that if $A$ and $B$ are bounded subsets of $\mathbb{R}$, then $A \cup B$ is a bounded subset of
\end{tcolorbox}

$\mathbb{R}$. Show that
$$\sup A \cup B = \max (\sup A, \sup B)$$

\begin{proof}
If $A$ and $B$ are bounded subsets of $\mathbb{R}$ then there exist real numbers $\sup A$ and $\sup B$. Let $x = \max (\sup A, \sup B)$.
$$\forall a \in A : a \leq \sup A \leq x$$
$$\forall b \in B : b \leq \sup B \leq x$$
Therefore:
$$\forall c \in A \cup B : c \leq x$$
Thus, $x$ is an upper bound of $A \cup B$. The same argument applies for the greatest lower bounds of $A$ and $B$. Thus, $A \cup B$ is bounded.

Now we prove that $x$ is the least upper bound. Suppose that there is another upper bound of $A \cup B$, namely $y$, such that $y<x$. Then $y$ must be an upper bound for both sets $A$ and $B$. Otherwise there would be an element $t$ in one of the sets $A$ or $B$ and, consequently, $A \cup B$, such that $t > y$. But this is impossible since $x$ is the least upper bound for at least one of the sets $A$ or $B$. Thus, $x = \max (\sup A, \sup B)$ is the least upper bound of the combined set $A \cup B$.
\end{proof}

\subsection*{Problem 4}

\begin{tcolorbox}
Fix $b>1$.

(a) If $m, n, p, q$ are integers, $n > 0$, $q > 0$, and $r = m/n = p/q$, prove that
\begin{equation} \label{eq:1}
(b^m)^{\frac{1}{n}} = (b^p)^{\frac{1}{q}}
\end{equation}
\end{tcolorbox}

\begin{proof}
We raise LHS and RHS of the expression \ref{eq:1} into $nq$-th power:
$$((b^m)^{\frac{1}{n}})^{nq} = b^{mq}$$
$$((b^p)^{\frac{1}{q}})^{nq} = b^{pn}$$
Notice that $\frac{m}{n}=\frac{p}{q} \Rightarrow pn = mq$. Hence, $b^{mq} = b^{pn}$.

We claim that if $a^n = b^n$ where $n \in \mathbb{Z}$, then $a=b$ by induction.
This claim is true for $n=1$:
$$a^1 = b^1 \Rightarrow a = b$$
For arbitrary $n \in \mathbb{N}$ we prove that $a^n = b^n$ if $a^{n-1} = b^{n-1}$:
$$a^n = b^n$$
$$ab^{-1}a^{n-1} = b^{n-1}$$
$$ab^{-1} = 1$$
$$a = b$$
Note that the claim is not true for $n=0$ since $a^0=1$ is always equal to $b^0=1$.

For negative integer powers the claim is also true. Consider $n \in \mathbb{Z}, n<0$. Assume $a^{-n} = b^{-n}$, then:
$$\frac{1}{a^n} = \frac{1}{b^n} \Rightarrow a^n = b^n \Rightarrow a=b$$

Going back to our example, $nq$ is an integer (product of two integers), therefore:
$$(b^m)^{\frac{1}{n}} = (b^p)^{\frac{1}{q}}$$
This means that there is only one rational number $b^r$ for $b>1$ and any given rational number $r$.
\end{proof}

\begin{tcolorbox}
(b) Prove that $b^{r+s} = b^r b^s$ if $r, s$ are rational.
\end{tcolorbox}

\begin{proof}
We know that $b^r+b^s = b^{r+s}$ for $r,s \in \mathbb{Z}$. We now prove that this is also the case for $r,s \in \mathbb{Q}$. Let $r=\frac{p}{q}$ and $s=\frac{m}{n}$. We raise LHS and RHS of the expression to the $qn$-th power:

$$\mbox{LHS: } (b^{\frac{p}{q}+\frac{m}{n}})^{qn} = (b^{\frac{pn+mq}{qn}})^{qn} = b^{pn+mq} $$

$$\mbox{RHS: } (b^{\frac{p}{q}}b^{\frac{m}{n}})^{qn}= ((b^p)^\frac{1}{q}(b^m)^\frac{1}{n})^{qn} = b^{pn} b^{mq} = b^{pn+mq}$$

LHS is equivalent to RHS, hence $(b^{r+s})^{qn} = (b^r b^s)^{qn} $ where $r, s \in \mathbb{Q}$ and $qn$ is integer. Therefore $b^{r+s} = b^r b^s$. 
\end{proof}

\begin{tcolorbox}
(c) If $x$ is real, define $B(x)$ to be the set of all numbers $b^t$, where $t$ is rational and $t \leq x$. Prove that
$$b^r = \sup B(r)$$
when $r$ is rational.
\end{tcolorbox}

\begin{proof}
First we prove that for $b>1 : t \leq r \Rightarrow b^t \leq b^r$. We note that $(b^n)^{-1} = \frac{1}{b^n} = b^{-n}$ for $n \in \mathbb{Q}$. Then $b^m \leq b^n \iff b^n b^{-m} \geq 1 \iff b^{n-m} \geq 1$. We can see that $n-m$ is a rational number. Let it be equal to $\frac{p}{q}$. Then $b^{n-m} = (b^p)^{\frac{1}{q}}$ is larger than or equal to $1$ as long as $n-m \geq 0$ (with equality being satisfied when $n=m$). Thus, $n-m \geq 0 \iff m \leq n \iff b^m \leq b^n$ for $b>1$. In other words, $f: t \mapsto b^t$ is a monotone (order preserving) map for $r \in \mathbb{Q}$.

Denote $R(r) = \{t \mid t \in \mathbb{Q}, t \leq r \}$, all rational numbers less than or equal to $r$. The monotone map $f$ maps the largest element of $R(r)$ to the largest element of $B(r)$, or $\forall t \in R(r): b^t \leq b^r$. Since all elements of $B(r)$ are less or equal to $b^r$, then it is an upper bound for $B(r)$. Since $b^r$ is an element of $B(r)$, it is the least upper bound.

\end{proof}

\begin{tcolorbox}
(d) Prove that $b^{x+y} = b^x b^y$ for all real $x$ and $y$.
\end{tcolorbox}

\begin{proof}
For all $m \in R(x)$ and $n \in R(y)$ ($m$ and $n$ are rational) the following holds by (b):
$$b^{m+n} = b^m b^n$$

Since map $t \mapsto b^t$ is monotone for $b>1$, $\forall m \in R(x) : b^m \leq \sup B(x)$ and $\forall n \in R(y) : b^n \leq \sup B(y)$:
$$b^{m+n} \leq \sup B(x) \sup B(y)$$
By definition $b^x = \sup B(x)$ and $b^y = \sup B(y)$, then:
$$\forall m \in R(x), \forall n \in R(y) : b^{m+n} \leq b^x b^y$$
We note that $x = \sup R(x)$ and $y = \sup R(y)$ by definition. Then $\sup \{ m+n \mid m \in R(x), n \in R(y)\} = x+y$, and:
$$\forall t \in B(x+y): t \leq b^x b^y$$

Thus, $b^x b^y$ is an upper bound for $B(x+y)$. We now prove that it is the least upper bound. Suppose there exists $r \in \mathbb{R}$, such that $r$ is an upper bound for $B(x+y)$ and $r < b^x b^y$. Then, since $\mathbb{Q}$ is dense in $\mathbb{R}$, there must exist rational $p \in R(x)$ and $q \in R(y)$ such that $r < b^p b^q < b^x b^y$. Then $r < b^p b^q = b^{p+q} \in B(x+y)$. Thus, $r$ cannot be an upper bound for $B(x+y)$. Contradiction.

Thus, $b^x b^y$ is the least upper bound for $B(x+y)$, or:
$$b^{x+y} = b^x b^y$$

\end{proof}

\subsection*{Problem 5}

\begin{tcolorbox}
Prove that no order can be defined in the complex field that turns it into an ordered
field.
\end{tcolorbox}

\begin{proof}
Suppose there exists an order that turns $\mathbb{C}$ into an ordered field. Then, the following must hold for any ordered field:
$$\forall a \in \mathbb{F} : a^2 \geq 0$$
Note that $-1$ cannot be equal to $0$ since only one distinct additive identity is possible in a field. Thus:
$$-1 = i^2 > 0$$
 The following must also hold for any ordered field:
$$ a > 0 \iff -a < 0$$
$$ -1 > 0 \iff 1 < 0$$
However the following must also hold:
$$1 = (-1)^2 > 0$$
Contradiction. Thus, no order can be defined in the complex field that turns it into an ordered field.

\end{proof}

\subsection*{Problem 6}

\begin{tcolorbox}
Suppose $z = a + bi$, $w = c + di$. Define $z < w$ if $a < c$ or $a = c, b < d$. Prove that this turns the set of all complex numbers into an ordered set. (This is known as a dictionary order, or lexicographic order)
\end{tcolorbox}

\begin{proof}
Consider $z = a + bi$, $w = c + di$. We can see that only one of the following can be true: $z<w$, $z>w$ or $z=w$:
\begin{itemize}
\item If $a > c$ then $z>w, z \neq w, z \not< w$;
\item If $a < c$ then $z<w, z \neq w, z \not> w$;
\item if $a = c$ and $b = d$ then $z \not> w, z = w, z \not< w$;
\item If $a = c$ and $b > d$ then $z > w, z \neq w, z \not< w$
\item if $a = c$ and $b < d$ then $z \not> w, z \neq w, z < w$
\end{itemize}

Consider $z = a + bi$, $w = c + di$, $u = e + fi$, such that $z<w, w<u$. Since $z<w$  then either $a<c$ or $a=c, b<d$. Since $w<u$  then either $c<e$ or $c=e, d<f$. Transitivity of the given order ($z<u$) follows from transitivity of standard ordering of real numbers $a,b,c,d,e,f$.
\end{proof}

\begin{tcolorbox}
Does this ordered set have the least-upper-bound
property? 
\end{tcolorbox}
Answer: No.

\begin{proof}
Consider the set $E(x) = \{ a+bi \mid a<x, b \in \mathbb{R} \}$ with the given order. All upper bounds of set $E(x)$ are $U = \{a+bi \mid a \geq x b \in \mathbb{R}\}$. However, for any upper bound $u = (a+bi) \in U$ there exists an upper bound $g \in U$ such that $g > u$, for example $g = a+(b+d)i$ where $d>0$. Therefore, no least upper bound exists for $E(x)$.
\end{proof}

\subsection*{Problem 7}

\begin{tcolorbox}
Prove that
\begin{equation} \label{eq:2}
|x + y|^2 + |x - y|^2 = 2|x|^2 + 2|y|^2    
\end{equation}
if $x \in \mathbb{R}^k$ and $y \in \mathbb{R}^k$. Interpret this geometrically, as a statement about parallelograms.
\end{tcolorbox}

\begin{proof}
We rewrite the expression using dot product:
$$(x+y) \cdot (x+y) + (x-y) \cdot (x-y) = 2 (x\cdot x) + 2 (y\cdot y)$$
We apply distributive property over addition:
$$ x \cdot x + 2(x \cdot y) + y \cdot y + x \cdot x - 2(x \cdot y) + y \cdot y = 2 (x \cdot x) + 2 (y\cdot y)$$
$$2 (x\cdot x) + 2 (y\cdot y) = 2 (x\cdot x) + 2 (y\cdot y)$$
Thus, the original expression is true.

To interpret this result geometrically, we consider a parallelogram in $\mathbb{R}^2$ with longer side $x$ and with shorter side $y$. Then $x+y$ is this parallelogram's longer diagonal and $x-y$ is its shorter diagonal. Therefore equality \ref{eq:2} is equivalent to the following observation: the sum of squares of lengths of parallelogram's diagonals is equal to twice the sum of squares of its (distinct) sides.

\end{proof}

\end{document}
