\documentclass{article}
\usepackage[utf8]{inputenc}
\usepackage[english]{babel}
\usepackage[]{amsthm}
\usepackage[]{amssymb} 
\usepackage{amsmath}
\usepackage{mathrsfs}
\usepackage{tcolorbox}


\newcommand{\R}{\mathbb{R}}
\newcommand{\N}{\mathbb{N}}
\newcommand{\norm}[1]{\left\lVert #1 \right\rVert}


\newtheorem{claim}{Claim}

\title{18.100B: Problem Set 10}
\author{Dmitry Kaysin}
\date\today

\begin{document}
\maketitle 


\subsection*{Problem 1}

\begin{tcolorbox}
Let $(f_n)$ be the sequence of functions on $\R$ defined as follows
\[ f_0(t) = \sin t \>\>\> \text{ and } \>\>\> f_{n+1}(t) = \frac{2}{3}f_n(t)+1 \>\>\> \text{ for $n \in \N$} \]
Show that $f_n \to 3$ uniformly on $\R$. What can you say if we choose $f_0(t) = t^2$?

Hint: Consider first the map $T(x) = \frac{2}{3}x + 1$ on $\R$.
\end{tcolorbox}

\begin{proof}

We first note that $T$ is a contraction mapping on $\R$, i.e. for any $a,b \in \R$:
\[ T(a) - T(b) = \frac{2}{3}a+1 - \frac{2}{3}b-1 = \frac{2}{3}(a-b) \]
Therefore, there exists a fixed point of $T$, denote it $L_t$.

Consider sequence of functions that is identical to $f_n(t)$ but with arbitrary starting element $x$:
\[ g_0(x) = x \quad \text{ and } \quad g_{n+1}(x) = T(g_n(x)) \quad \text{ for $n \in \N$} \]
Sequence $g_n(x)$ must converge to $L_x$:
\[ \lim g_n(x) = L_x = T(L_x) \]
Furthermore:
\begin{gather*}
    \lim f_{n}(x) = \lim f_{n+1}(x) = \lim \left( \frac{2}{3}f_n(x)+1 \right) \\
    L_x = \frac{2}{3} L_x +1 \\
    L_x = 3    
\end{gather*}
Therefore, $g_n(x)$ converges to $3$ pointwise.

We can see that $g_n$ is monotone by considering the difference between subsequent elements of sequence $g_n$:
\[ g_{n+1}(x) - g_n(x) = \frac{2}{3}g_n(x) + 1 - g_n(x) = \frac{3-g_n(x)}{3} \]
Specifically, $g_n$ is monotonically increasing in case $g_n(0) < 3$ and monotonically decreasing if $g_n(0) > 0$.
Each $g_n$ is a polynomial, thus continuous.
Limit function $g(x)=3$ is also continuous.
From this we conclude that sequence $g_n$ converges uniformly if domain of each $g_n$ is compact.

If range of $g_n$ is compact, then domain of $g_{n+1}$ must be also compact since $g_n$ is continuous.
By induction, if range of $g_0$ is compact, domains of all $g_n$ are also compact and thus $g_n$ converges uniformly.

This is the case for $g_n(\sin t)$, therefore $g_n(\sin t) \to 3$ uniformly.
However, this is not the case for $g_n(t^2)$ since range of $g_n(t^2) = t^2$ is $[0,\infty)$, thus not compact.

\end{proof}


\subsection*{Problem 2}

\begin{tcolorbox}
Suppose $\varphi : [0, \infty) \to \R$ is continuous and satisfies
\[ 0 \leq \varphi(t) \leq \frac{t}{2+t} \quad (t \geq 0)\]
Define the sequence $(f_n)$ by setting $f_0(t) = \varphi(t)$ and $f_{n+1}(t) = \varphi(f_n(t))$ for $t \geq 0$ and $n \in \N$.
Prove that the series $F(t) = \sum_{n=0}^{\infty} f_n(t)$ converges for every $t \geq 0$ and that $F$ is continuous on $n=[0, \infty)$.
\end{tcolorbox}

\begin{proof}

We first note that $f_n(t)$ is decreasing:
\[
\frac{f_{n+1}(t)}{f_n(t)} = \frac{\varphi(f_n(t))}{f_n(t)} \leq \frac{f_n(t)}{f_n(t)(2+f_n(t))} = \frac{1}{2+f_n(t)} \leq \frac{1}{2} 
\]
and 
\[ f_{n+1}(t) \leq \frac{1}{2} f_n(t) \]
We also note that $f_1(t) \leq 1$, therefore:
\[ f_{n+1}(t) \leq \frac{1}{2^n} f_1(t) \leq \frac{1}{2^n} \]

We apply Weierstrass M-test (Rudin 7.10) to examine convergence of the series $s(t) = \sum_{n=0}^{\infty} f_n(t)$:
\[
f_{n+1}(t) \leq \frac{1}{2^n} \text{ and $\sum_{n=0}^{\infty} \frac{1}{2^n} \to 0$ as $n \to \infty$}
\]
We conclude that $s(t) \to F(t)$ uniformly.

Since $\varphi(t)$ is continuous each $f_n(t)$ must be continuous; thus partial sums of $s(t)$ must also be continuous.
Sequence of continuous functions converges uniformly to a continuous function.
Thus, $F(t)$ is continuous.

\end{proof}


\subsection*{Problem 3}

\begin{tcolorbox}
Does $f(t) = \sum_{k=1}^{\infty} \sin^2 \left( \frac{t}{k} \right)$ define a differentiable function on $\R$?
\end{tcolorbox}

\begin{proof}

Choose an arbitrary closed interval $[a,b] \subseteq \R$ and denote $s = \max(|a|,|b|)$.
Denote $g_k(t) = \sin^2 \left( \frac{t}{k} \right)$. First we notice that for each $g_k(t)$ on $[a,b]$:
\[ |g_k(t)| = \left| \sin \left( \frac{t}{k} \right)^2 \right| \leq \left( \frac{t}{k} \right)^2 \leq \frac{s^2}{k^2} \]
Here we use the fact that $|\sin x| \leq |x|$.
Series $\sum_{k=1}^{\infty} \frac{s^2}{k^2} = s^2 \sum_{k=1}^{\infty} \frac{1}{k^2}$ converges, therefore $f_k(t)$ converges pointwise on $[a,b]$.

Function $g_k$ is differentiable on $[a,b]$ for any $k$:
\[ g'_k(t) = \frac{2}{k} \sin \left( \frac{t}{k} \right) \cos \left( \frac{t}{k} \right) \]
Thus any partial sum of $f_k$ is also differentiable on $[a,b]$.

We notice that for each $g'_k(t)$ on $[a,b]$:
\[
\left| \frac{2}{k} \sin \left( \frac{t}{k} \right) \cos \left( \frac{t}{k} \right) \right| \leq \left| \frac{2t}{k^2} \right| \leq \frac{2s}{k^2}
\]
Series $\sum_{k=1}^{\infty} \frac{s}{k^2} = s \sum_{k=1}^{\infty} \frac{1}{k^2}$ converges, therefore $\sum g'_k(t)$ converges uniformly on $[a,b]$.

By Rudin 7.17 we have that $f$ is differentiable on $[a,b]$. Therefore, $f$ is differentiable on $\R$.

\end{proof}


\subsection*{Problem 4}

\begin{tcolorbox}
Suppose $(f_n)$ is a sequence of continuous functions such that $f_n \to f$ uniformly on a set $E$. Prove that
\[ \lim_{n \to \infty} f_n(x_n) = f(x) \]
for every sequence of points $x_n \in E$ such that $x_n \to x$, and $x \in E$. Is the converse of this true?
\end{tcolorbox}

\begin{proof}

Fix $\epsilon>0$. Since $f_n \to f$ uniformly, there must exist $M$ such that for all $m>M$ for any $z \in E$:
\[ \norm{ f_m(z) - f(z) } \leq \frac{\epsilon}{2} \]

By Rudin 7.12 $f$ must be continuous. Since $f$ is continuous, and $x_n \to x$ there exists $K$ such that for all $k \geq K$:
\[ \norm{ f(x_k) - f(x) } \leq \frac{\epsilon}{2} \]

Combining the above:
\[ \norm{ f_m(x_k) - f(x) } \leq \norm{ f_m(x_k) - f(x_k) } + \norm{ f(x_k) - f(x) } \leq \epsilon \]

Taking $N = \max(M,K)$ we have that for all $n \geq N$:
\[ \norm{ f_n(x_n) - f(x) } \leq \epsilon, \]
which proves that $\lim_{n \to \infty} f_n(x_n) = f(x)$.

\end{proof}

Now consider the converse. As we will see, the fact that for every convergent sequence $x_n \to x$ the following holds:
\[ \lim_{n \to \infty} f_n(x_n) = f(x), \]
it does not necessarily imply that convergent sequence of continuous functions $f_n$ converges uniformly, even if the limit function of $f_n$ is continuous.

\begin{proof}

Counterexample: Consider sequence of functions $\frac{x^2}{n}$, which converges to $0$.

Take arbitrary convergent sequence $x_n \to x$ with $x_n,x \in \R$. Fix $\epsilon>0$. Since $x_n \to x$ there exists $M$ such that for all $m \geq M$:
\begin{gather*}
    \norm{ x_m - x } \leq \epsilon \\
    \norm{ x_m } \leq \norm{ x } \norm{ + \epsilon }
\end{gather*}
We will now prove that $f_n(x_n) \to f(x)$:
\[ \norm{ f_m(x_m) - f(x) } = \norm{ \frac{x_m^2}{m} - 0 } \leq \frac{(\norm{ x } + \epsilon)^2}{m}. \]
The last expression can be set arbitrarily close to $0$ by choosing sufficiently high $m$.

However, it is easy to see that convergence of $\frac{x^2}{n}$ is not uniform on $\R$.

\end{proof}


\subsection*{Problem 5}

\begin{tcolorbox}
Suppose $(f_n)$ is a sequence of real-valued functions that are Riemann-integrable on all compact subintervals of $[0,\infty)$. Assume further that:

a) $f_n \to 0$ uniformly on every compact subset of $[0,\infty)$;

b) $0 \leq f_n(t) \leq e^{-t}$ for all $t \geq 0$ and $n \in \N$.

Prove that
\[ \lim_{n \to \infty} \int_0^\infty f_n(t) dt = 0, \]

where the improper integral $\int_0^\infty f_n(t) dt$ is defined as  $\lim_{b\to\infty} \int_0^b f_n(t) dt$.
\end{tcolorbox}

\begin{proof}

Fix sufficiently small $\epsilon \in (0,1)$. Denote $\epsilon_s = \sqrt{\epsilon}$. Since $f_n \to 0$ uniformly, there exists $M$ such that for all $m \geq M$:
\[ | f_m(t) - 0 | \leq \epsilon_s \]

Consider point $x$ such that $e^{-x} = \epsilon_s$:
\[ x = - \ln(\epsilon_s) \]

We notice that
\[
\left| \int_0^\infty f_m \right| 
= \left| \int_0^x f_m + \int_x^\infty f_m \right|
\leq \left| \int_0^x f_m \right| + \left| \int_x^\infty f_m \right|
\]

We evaluate the first summand:
\[
\left| \int_0^x f_m(t) dt \right| \leq | x \epsilon_s | 
= | - \ln(\epsilon_s) \epsilon_s | 
=  - \ln(\epsilon_s) \epsilon_s
\]

We evaluate the second summand. Since $f_n(t) \leq e^{-t}$:
\[
\left| \int_x^\infty f_m \right| 
\leq \left | \int_x^\infty e^{-t} dt \right|
= \left| \int_{- \ln(\epsilon_s)}^\infty e^{-t} dt \right|
= \left| 0 - e^{ - \ln \epsilon_s} \right| = \epsilon_s
\]

We conclude that 
\[
\left| \int_0^\infty f_m \right| 
\leq - \ln(\epsilon_s) \epsilon_s + \epsilon_s
= \epsilon_s (1-\ln(\epsilon_s))
\]
We notice that for $0 < \epsilon < 1$:
\[ \epsilon_s (1-\ln(\epsilon_s)) < \epsilon_s^2 < \epsilon \]

Therefore $\int_0^\infty f_n \to 0$ as $n \to \infty$.

\end{proof}

\begin{tcolorbox}
Moreover, give an explicit example for a sequence $(f_n)$, so that condition b) does not hold and the conclusion above fails.
\end{tcolorbox}

Answer: $g_n(t) = \frac{1}{n(x+1)}$.

\begin{proof}

Sequence of functions $g_n$ is Riemann-integrable on all compact subintervals of $[0,\infty)$ and converges uniformly to $0$.
However each improper integral $\int_0^\infty g_n$ diverges, therefore the conclusion of $\int_0^\infty g_n$ having a limit as $n \to \infty$ fails.

\end{proof}


\subsection*{Problem 6}

\begin{tcolorbox}
Suppose $(f_n)$ is an equicontinuous sequence of functions on a compact set $K$, and $f_n \to f$ pointwise on $K$.
Prove that $f_n \to f$ uniformly on $K$.
\end{tcolorbox}

\begin{proof}

Fix $\epsilon>0$. Since $f_n$ is equicontinuous, there exists $\delta>0$ such that for all $n \in \N$ and $a, b \in K$ with $d(a,b)<\delta$ we have:
\begin{equation}
    \norm{ f_n(a) - f_n(b) } < \frac{\epsilon}{3}
\end{equation}
We also note that:
\begin{equation}
    \norm{ f(a) - f(b) }
    = \norm{ \lim_{n\to\infty} f_n(a) - \lim_{n\to\infty} f_n(b) } 
    = \lim_{n\to\infty} \norm{ f_n(a) - f_n(b) }
    \leq \frac{\epsilon}{3}
\end{equation}

Consider open cover of $K$ by open balls $B_\delta(x)$ with center $t$ and radius $\delta$. Denote such cover $\mathcal{U}_t$ (indexed over center points of $\delta$-balls).
Since $K$ is compact, there exists finite subcover of $\mathcal{U}_t$; denote it $\mathcal{U}_{t_k}$. Since $f_n \to f$ pointwise, for each $t_k$ we can find $N_k$ such that for all $n_k \geq N_k$ we have:
\[ \norm{ f_{n_k}(t_k) - f(t_k) } < \frac{\epsilon}{3} \]

Since there are finitely many $t_k$, we can select the largest $N_k$, denote it $M$. For each $m \geq M$ we have:
\begin{equation}
    \norm{ f_m(t_k) - f(t_k) } < \frac{\epsilon}{3}
\end{equation}
Consider arbitrary point $x \in K$.
It must belong to at least one of the sets from $\mathcal{U}_k$.
Choose one such open $\delta$-ball with center $t$, then we have:
\[ \norm{ x - t } < \delta \]
By (1) we have:
\[ \norm{ f_m(x) - f_m(t) } < \frac{\epsilon}{3} \]
By (2) we have:
\[ \norm{ f(t) - f(x) } < \frac{\epsilon}{3} \]
By (3) we have:
\[ \norm{ f_m(t) - f(t) } < \frac{\epsilon}{3} \]
Finally, we notice that:
\[
\norm{ f_m(x) - f(x) } 
\leq
\norm{ f_m(x) - f_m(t) } 
+ \norm{ f(t) - f(x) } 
+ \norm{ f_m(t) - f(t) }
<
\epsilon
\]
Therefore, $f_n \to f$ uniformly.

\end{proof}


\subsection*{Problem 7}

\begin{tcolorbox}
Show that any uniformly bounded sequence of differentiable functions on a compact interval with uniformly bounded derivatives has a convergent subsequence.
\end{tcolorbox}

Denote sequence of functions $(f_n)$ over compact interval $K$. Suppose $|f'_n(x)| \leq M$.

\begin{proof}

We have the result by the Arzela-Ascoli theorem (Rudin 7.25) if the following conditions hold:\\
1) Each $f_n$ is continuous;\\
2) Each $f_n$ is pointwise bounded;\\
3) Sequence $(f_n)$ is equicontinuous.

Condition 1 holds since each $f_n$ is differentiable, thus continuous. Condition 2 holds since sequence $(f_n)$ is uniformly bounded.

Next we will prove condition 3. Fix $\epsilon>0$. Consider two arbitrary points $a,b \in K, b > a$ such that $d(a,b) < \frac{\epsilon}{M}$. By Mean Value Theorem there exists $c \in K$ such that:
\[ f_n'(c) = \frac{f_n(b) - f_n(a)}{b - a}, \]
from which we have:
\[ f_n(b) - f_n(a) = (b - a) f_n'(c) < \frac{\epsilon}{M} M = \epsilon \]
We conclude that sequence $(f_n)$ is equicontinuous. Thus, $(f_n)$ has a convergent subsequence.

\end{proof}


\end{document}
