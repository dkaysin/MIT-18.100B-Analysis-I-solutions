\documentclass{article}
\usepackage[utf8]{inputenc}
\usepackage[english]{babel}
\usepackage[]{amsthm}
\usepackage[]{amssymb} 
\usepackage{amsmath}
\usepackage{tcolorbox}
\usepackage{mathtools}

\newcommand{\R}{\mathbb{R}}
\newcommand{\N}{\mathbb{N}}
\DeclarePairedDelimiter{\norm}{\lVert}{\rVert}
\DeclarePairedDelimiter{\abs}{\lvert}{\rvert}
\DeclarePairedDelimiter{\set}{ \{ }{ \} }

\title{18.100B: Problem Set 5}
\author{Dmitry Kaysin}
\date{November 2019}
\begin{document}
\maketitle 


\subsection*{Problem 1}

\begin{tcolorbox}
Let $\mathcal{M}$ be a complete metric space, and let $X \subseteq \mathcal{M}$.
Show that $X$ is complete if and only if $X$ is closed.
\end{tcolorbox}

\begin{proof}

Suppose $X$ is complete.
Any limit point of $X$ is the limit of some Cauchy sequence in $X$.
Since any Cauchy sequence converges in $X$, its limit must be in $X$.
Thus all limit points of $X$ are in $X$ and therefore $X$ must be closed.

Now suppose $X$ is closed.
Any Cauchy sequence in $\mathcal{M}$ that is also in $X$ must converge to some limit in $\mathcal{M}$.
Limit of any Cauchy sequence in $X$ must be either a point of $X$ or a limit point of $X$. Since $X$ is closed, it contains all its limit points.
Therefore, all Cauchy sequences in $X$ converge in $X$.
Thus, $X$ is a complete metric space. 

\end{proof}


\subsection*{Problem 2}

\begin{tcolorbox}
a) Show that a sequence in an arbitrary metric space $\set{x_n}$ converges if and only if the ‘even’ and ‘odd’ subsequences $\set{x_{2n}}$ and $\set{x_{2n-1}}$ both converge to the same limit.
\end{tcolorbox}

\begin{proof}

Suppose sequence $\set{x_n}$ converges to $x$.
Then for any $\epsilon$ we can find $N \in \N$ such that for all $n > N: d(x, x_n) < \epsilon$.
Clearly, $d(x, x_{2n}) < \epsilon$ and $d(x, x_{2n-1}) < \epsilon$.
Therefore subsequence $\set{x_{2n}}$ and subsequence $\set{x_{2n-1}}$ must converge to $x$.

Suppose both $\set{x_{2n}}$ and $\set{x_{2n-1}}$ converge to $x$.
Then for any $\epsilon$ we can find $N, M \in \N$ such that for all $n > N: d(x, x_{2n}) < \epsilon$ and for all $n > M: d(x, x_{2n-1}) < \epsilon$.
We notice that for all $n > 2N: d(x, x_n) < \epsilon$.
Therefore, sequence $\set{x_n} $must converge to $x$.

\end{proof}

\begin{tcolorbox}
b) Show that a sequence in an arbitrary metric space $\set{x_n}$ converges if and only if the subsequences $\set{x_{2n}}$, $\set{x_{2n-1}}$, and $\set{x_{5n}}$ all converge.
\end{tcolorbox}

\begin{proof}

Consider $\set{x_{n}}$ and its subsequence $\set{x_{f(n)}}$ where $n$ and $f(n)$ are natural numbers and $f(n) \geq n$.
Suppose $\set{x_{n}}$ converges to $x$.
Then for any $\epsilon > 0$ there exists $N$ such that for all $n > N: d(x, x_n) < \epsilon$.
For any given $n$ we notice that $n>N \Rightarrow f(n) 
geq n > N$.
Therefore $d(x, x_{f(n)}) < \epsilon$ and thus subsequence $\set{x_{f(n)}}$ converges to $x$.

Conversely, suppose that $\set{x_{f(n)}}$ converges to $x$.
Then for any $\epsilon > 0$ there exists $N$ such that for all $n: f(n) > N: d(x, x_{f(n)}) < \epsilon$.
We notice that for any $N$ there exists $K$ such that $K > f(N)$ by the Archimedean property.
Then $K > f(N) > N$.
Therefore for all $n > K : d(x, x_n) < \epsilon$ and thus sequence $\set{x_n}$ converges to $x$.

We conclude that sequence converges to $x$ if and only if all its subsequences converge to $x$.
Case in hand is a special case of this general theorem for three chosen subsequences.

\end{proof}


\subsection*{Problem 3}

\begin{tcolorbox}
If $\set{x_n}$ and $\set{y_n}$ are two bounded sequences of real numbers, show that:
\begin{itemize}
    \item $\lim \sup (x_n + y_n) \leq \lim \sup x_n + \lim \sup y_n$;
    \item $\lim \inf (x_n + y_n) \geq \lim \inf x_n + \lim \inf y_n$.
\end{itemize}
\end{tcolorbox}

\begin{proof}

Let $x = \lim \sup x_n, \>\> y = \lim \sup y_n$.
Since sequences $x_n$ and $y_n$ are bounded, $(x_n+y_n)$ must be also bounded, thus we can find a convergent subsequence
$s_{n_k} = (x_{n_k} + y_{n_k}); \>\> s_{n_k} \to s$.
Since sequence $x_{n_k}$ is bounded, there must exist its subsequence that is convergent; denote it $x_{n_a} \to a$.
Now consider a bounded sequence $y_{n_a}$, which must contain a convergent subsequence $y_{n_b} \to b$.
We note that $x_{n_b}$ is a subsequence of convergent sequence, thus $x_{n_b} \to a$.
Now consider sequence $(x_{n_b}+y_{n_b})$, which must converge to $a+b$.
At the same time, $(x_{n_b}+y_{n_b})$ is a subsequence of convergent sequence $s_{n_k}$.
Therefore $s_{n_k}$ must also converge to $a+b$, which means $s=a+b$.
We notice that $a \leq x$ and $b \leq y$.
Therefore:
\[ s \leq x + y \]
We can see that limit of any convergent subsequence of $(x_n+y_n)$ must be strictly less than $x+y$.
In other words, $x+y$ is an upper bound of the set of subsequential limits of $s_n$.
Supremum of the set of subsequential limits of $s_n$ is its least upper bound, therefore:
\[ \lim \sup (x_n + y_n) \leq \lim \sup x_n + \lim \sup y_n. \]
Similar logic can be used to show that
\[ \lim \inf (x_n + y_n) \geq \lim \inf x_n + \lim \inf y_n. \]

\end{proof}

\begin{tcolorbox}
Moreover, show that if $\set{x_n}$ converges, then both inequalities are actually equalitites.
\end{tcolorbox}

\begin{proof}

Suppose $\set{x_n}$ converges to $p$.
Then its every subsequence converges to $p$.
For $\lim \sup \set{y_n} = a$ there must exist a subsequence $\set{y_{n_a}}$ of $\set{y_n}$ that converges to $a$.
Consider subsequence $\set{ x_{n_a} + y_{n_a} }$ of the sequence $\set{ x_n + y_n }$ for which:
\[ \lim \sup \set{ x_{n_a} + y_{n_a} } \leq \lim \sup \set{ x_n + y_n } \]
We note that  subsequence $\set{ x_{n_a} + y_{n_a} }$ converges to $p+a$, therefore:
\[ p + a \leq \lim \sup \set{ x_n + y_n } \]
\[ \lim \sup \set{ x_n } + \lim \sup \set{y_n} \leq \lim \sup \set{ x_n + y_n } \]
At the same time, using the result from the first part of the problem we can see that
\[ \lim \sup \set{ x_n + y_n } \leq \lim \sup \set{ x_n } + \lim \sup \set{y_n} \]
From this follows the equality
\[ \lim \sup \set{ x_n + y_n } = \lim \sup \set{ x_n } + \lim \sup \set{y_n} \]

Using the same logic we can prove that for convergent $\set{ x_n }$ and bounded $\set{ y_n }$ that
\[ \lim \inf \set{ x_n } + \lim \inf \set{ y_n } = \lim \inf \set{ x_n + y_n} \]

\end{proof}


\subsection*{Problem 4}

\begin{tcolorbox}
The ‘sequence of averages’ of a sequence of real numbers $\set{x_n}$ is the sequence $\set{a_n}$ defined by
\[ a_n = \frac{x_1+x_2+\dots+x_n}{n} \]
If $\set{x_n}$ is a bounded sequence of real numbers, then show that
\[ \lim \inf x_n \leq \lim \inf a_n \leq \lim \sup a_n \leq \lim \sup x_n. \]
\end{tcolorbox}

\begin{proof}

Denote $x^* = \lim \sup x_n$.
For a given $\epsilon>0$ consider $K = \set{ k \in \N : x_k \geq x^*+\epsilon }$ . Suppose $K$ is infinite. 
Then bounded sequence $\set{ x_k }_{k \in K}$ must contain a convergent subsequence $x_{k_c} \to c$.
Limit of $x_{k_c}$ must lie in the closed interval:
\[ \lim \inf x_{k_c} \leq c \leq \lim \sup x_{k_c} \]
Since $x^*+\epsilon < \lim \inf x_{k_c}$ we've found a subsequence of $x_n$ that converges to a point that is strictly greater than $\lim \sup x_n$. Contradiction.
Therefore, $K$ must be finite.

Define $\mathcal{S}_n = \set{ i \in \N : i \in K, i \leq n }$ and $\mathcal{T}_n = \set{ i \in \N : i \notin K, i \leq n }$ and define sequences $s_n$ and $t_n$ by:
\[ s_n = \sum_{i \in \mathcal{S}_n}x_i, \>\>\> t_n = \sum_{i \in \mathcal{T}_n}x_i \]

Since $\mathcal{S}_n \cup \mathcal{T}_n = \set{ i \in \N : i \leq n }$, the set of the first $n$ natural numbers, $s_n+t_n$ is equal to the sum of first $n$ elements of the sequence $x_n$.
Therefore:
\[ a_n = \frac{s_n}{n}+\frac{t_n}{n}. \]
We notice that $\frac{s_n}{n} \to 0$ since $s_n \to s$, some finite number, and $n \to \infty$.
Furthermore, we notice that every element of $\set{ x_k }_{k \in \N, k \notin K}$ is less than $x^* + \epsilon$.
Then the sum of any $n$ elements from $\set{ x_k }_{k \in \N, k \notin K}$ is less than $n(x^* + \epsilon)$ for any $n \in \N$.
Therefore:
\[ \frac{t_n}{n} < x^* + \epsilon \]
Using the result from problem 3 we find that:
\[ \lim \sup (\frac{s_n}{n}+\frac{t_n}{n}) \leq \lim \sup \frac{s_n}{n} + \lim \sup \frac{t_n}{n} \]
Since $\lim \sup \frac{s_n}{n} = 0$ and $\lim \sup \frac{t_n}{n} \leq x^* + \epsilon$:
\[ \lim \sup a_n \leq x^* + \epsilon \]
We notice that $\lim \sup x_n$ is a lower bound for the set $\set{ x^*+\epsilon : \epsilon>0 }$, therefore it cannot be greater than $x^*$.
Thus $\lim \sup x_n \leq x^*$.
We conclude that:
\[ \lim \sup a_n \leq \lim \sup x_n \]

Similar logic can be employed to show that $\lim \inf x_n \leq \lim \inf a_n$.

\end{proof}

\begin{tcolorbox}
In particular, if $x_n \to x$ then show that $a_n \to x$.
\end{tcolorbox}

\begin{proof}

If $x_n \to x$ then $\lim \inf x_n = \lim \sup x_n = x$.
Therefore $\lim \inf a_n = \lim \sup a_n = x$.
From this $a_n \to x$.

\end{proof}

\begin{tcolorbox}
Does the convergence of $\set{a_n}$ imply the convergence of $\set{x_n}$?
\end{tcolorbox}

Answer: Convergence of $a_n$ does not necessarily imply the convergence of $x_n$.

\begin{proof}

Counterexample: Consider sequence $\set{ x_n } = \set{ (-1)^n : n \in \N }$. Sequence $a_n$ for such $x_n$ converges to $0$. Indeed $a_n$ can be rewritten as:
\[ a_n = \frac{0 \cdot k + (-1)^{n-2k}}{n} = \frac{0}{n} + \frac{(-1)^{n-2k}}{n} \]
where $k = \left \lfloor \frac{n}{2} \right \rfloor$.
From this it is clear that $\frac{(-1)^{n-2k}}{n} \to 0$ as $n \to 0$. At the same time, $x_n$ clearly does not converge.

\end{proof}


\subsection*{Problem 5}

\begin{tcolorbox}
Consider any sequence $(x_n)$ defined by choosing $0 < x_1 < 1$ and then defining $x_{n+1} = 1-\sqrt{1 - x_n}$ for $n \geq 0$.
Show that $x_n$ is a decreasing sequence converging to zero.
\end{tcolorbox}

\begin{proof}

Suppose that $0 < x_n < 1$. Then
\begin{gather*}
    0<1-x_n<1 \\
    0<\sqrt{1-x_n}<1 \\
    0<1-\sqrt{1-x_n}<1 \\
    0<x_{n+1}<1
\end{gather*}

By induction with $0 < x_1 < 1$ we conclude that all elements of sequence $x_n$ are between $0$ and $1$.

Furthermore, $x_n$ is decreasing.
To prove this, consider the difference:
\[ d = x_{n+1} - x_n = 1-\sqrt{1-x_n} - x_n = 1-x_n - \sqrt{1-x_n} = \sqrt{1-x_n} \left(\sqrt{1-x_n}-1\right) \]
We notice that $0 < \sqrt{1-x_n} < 1$, then $d < 0$, thus for any $n\in\N: x_{n+1} < x_n$.
Therefore, limit of $x_n$ must lie in $[0,1)$.

Suppose that $\lim x_n = L > 0$.
Then for all $x_n$:
\begin{gather*}
    x_n > L \\
    x_{n+1} > L \\
    1-\sqrt{1-x_n} > L \\
    \sqrt{1-x_n} < 1-L \\
    1-x_n < (1-L)^2 \\
    x_n-1 > -(1-L)^2 \\
    x_n > 1-(1-L)^2   
\end{gather*}

We then examine the difference
\[ 1-(1-L)^2 - L = (1-L) - (1-L)^2 = (1-L)(1-1+L) = L(1-L), \]
which is greater than $0$ for any $0<L<1$. Therefore
\[ x_n > 1-(1-L)^2 > L \]
However, we then can find $\epsilon$-neighbourhood of $L$ that contains no points of $x_n$.
Therefore $L>0$ cannot be a limit of $x_n$.
Contradiction.
We conclude that $\lim x_n = 0$.

Another way to see that the limit of sequence $x_n$ is $0$ is to evaluate its first few elements:
\begin{align*}
    x_2 & = 1-\sqrt{1-x_1} = 1-(1-x_1)^{\left(\frac{1}{2}\right)^1} \\
    x_3 & = 1-\sqrt{1-x_2} = 1-\sqrt{1-\left(1-\sqrt{1-x_1}\right)} = 1-(1-x_1)^{\left(\frac{1}{2}\right)^2}
\end{align*}
    
It is easy to see that:
\[ x_n = 1-(1-x_1)^{\left(\frac{1}{2}\right)^n} \]
For $0 < x_1 <1$ we notice that $(1-x_1)^p \to 1$ as $p \to 0$.
Therefore $x_n \to 0$ as $n \to \infty$.

\end{proof}

\begin{tcolorbox}
Also, show that $\frac{x_{n+1}}{x_n} \to \frac{1}{2}$.
\end{tcolorbox}

\begin{proof}

\[ \frac{x_{n+1}}{x_n} = \frac{1-\sqrt{1-x_n}}{x_n} = \frac{x_n}{x_n \left( 1+\sqrt{1-x_n} \right) } = \frac{1}{1+\sqrt{1-x_n}} \]

As $x_n \to 0$ we can see that limit of the denominator of the above expression is $2$. Thus, $\frac{x_{n+1}}{x_n} = \frac{1}{2}$.

\end{proof}


\subsection*{Problem 6}

\begin{tcolorbox}
The Greeks thought that the number $\Phi$, known as the Golden Mean, was the ratio of the sides of the most aesthetically pleasing rectangles.
Imagine a line segment $A$ divided into two smaller line segments $B$ and $C$, with lengths $a$, $b$, and $c$ respectively and $b > c$.
If the proportion between $a$ and $b$ is the same as the proportion between $b$ and $c$, then we call this proportion $\Phi$.

a) Show that with $a, b, c$ as above, $\Phi = \frac{b}{c}$ satisfies $\Phi^2 = \Phi + 1$.
Conclude that $\Phi = \frac{1+\sqrt{5}}{2}$
\end{tcolorbox}

\begin{proof}

\begin{gather*}
    \Phi = \frac{a}{b} = \frac{b}{c} \\
    ac=b^2 \\
    ac^2 = b^2 c \\
    \frac{b^2}{c^2} = \frac{a}{c} \\
    \frac{b^2}{c^2} = \frac{b+c}{c} \\
    \frac{b^2}{c^2} = \frac{b}{c} + 1 \\
    \Phi^2 = \Phi + 1
\end{gather*}

Solving for $\Phi$ we get:
\[ \Phi = \frac{1 \pm \sqrt{5}}{2} \]
Since $\Phi$ is a ratio between lengths of line segments, it cannot be negative.
Thus, $\Phi = \frac{1 + \sqrt{5}}{2}$.

\end{proof}

\begin{tcolorbox}
b) Show that:
\[ \Phi = 1+\frac{1}{1+\frac{1}{1+\frac{1}{1+\dots}}} \]
\end{tcolorbox}

\begin{proof}

Consider sequence $x_n$, which is defined by a recurrence relation:
\[ x_1 = 1, \>\>\>\>\> x_{n+1} = 1+\frac{1}{x_n} \]
Sequence $x_n$ is positive and bounded.
Indeed, for any $n \in \N$: $1 \leq x_n \leq 2 $.

Now consider two subsequences of $x_n$: odd $x^o_n$ and even $x^e_n$:
\begin{align*}
    x^o_n & = x_n, \quad\>\>\> x^o_{n+1} = 1+\frac{1}{1+\frac{1}{x_n}} \\
    x^e_n & = x_{n+1}, \>\>\> x^e_{n+1} = 1 + \frac{1}{1+\frac{1}{x_{n+1}}}
\end{align*}
We examine difference between two subsequent elements of subsequence $x^o_{n}$:
\begin{multline} \label{eq:1}
    x^o_{n+1} - x^o_{n} 
    = 1 + \frac{1}{1+\frac{1}{x_n}} - x_n 
    = - \frac{x_n^2 - x_n - 1}{x_n + 1} \\
    = -\frac{(x_n-\Phi)(x_n-\frac{1-\sqrt{5}}{2})}{x_n+1} 
    = \frac{(\Phi-x_n)(x_n+\frac{\sqrt{5}}{2}-1)}{x_n+1}    
\end{multline}

We notice that expression $\ref{eq:1}$ is positive only if $x_n < \Phi$.
Furthermore we claim that if $x^o_n < \Phi$ then $x^o_{n+1} < \Phi$.
To prove that we consider the difference 
\begin{multline*}
    x^o_{n+1} - \Phi = \frac{2x_n+1}{x_n+1} - \frac{1+\sqrt{5}}{2} \\
    = \frac{4x_n+2-(x+\sqrt{5}x_n + 1 + \sqrt{5})}{2(x_n+1)}
    = \frac{x_n(3-\sqrt{5})+(1-\sqrt{5})}{2(x_n+1)} \\
    = \frac{4x_n + 3 + \sqrt{5} - 3\sqrt{5}-5}{2(x_n+1)(3+\sqrt{5})} = \frac{2x_n-1-\sqrt{5}}{(x_n+1)(3+\sqrt{5})},    
\end{multline*}
which is positive if and only if $x_n < \Phi$.
Therefore $x^o_n < \Phi \Rightarrow x^o_{n+1} < \Phi$.
We notice that $x^o_1 = 1 < \Phi$, therefore the whole sequence $x^o_n$ is increasing.

Moreover, subsequence $x^o_n$ is bounded by boundedness of $x_n$; therefore, it must converge.
Any convergent sequence in complete metric space is a Cauchy sequence.
Therefore expression $\ref{eq:1}$ must be equal to zero as $n \to \infty$.
\[
    \lim_{n \to \infty} \frac{(\Phi-x_n)(x_n+\frac{\sqrt{5}}{2}-1)}{x_n+1} 
    = \frac{(\Phi-\lim x^o_n)(\lim x^o_n+\frac{\sqrt{5}}{2}-1)}{\lim x^o_n+1} = 0,
\]
which is true if and only if $\lim x^o_n = \Phi$.

We can prove in a similar way that subsequence $x^e_n$ also converges to $\Phi$.

Since odd and even subsequences of sequence $x_n$ converge to $\Phi$, $x_n$ itself must converge to $\Phi$.

\end{proof}

\begin{tcolorbox}
c) Show that:
\[ \Phi = \sqrt{1+\sqrt{1+\sqrt{1+\sqrt{1+\sqrt{1+\dots}}}}} \]
\end{tcolorbox}

\begin{proof}
Consider sequence $y_n$, which is defined by a recurrence relation:
\[ y_1 = 1, \quad\quad y_{n+1} = \sqrt{1 + y_n} \]
\end{proof}

We consider the ratio of two subsequent elements of $y_n$:
\[ \frac{y_{n+1}}{y_n} = \frac{\sqrt{1+y_n}}{y_n} \]
\begin{multline*}
    y_{n+1} > y_n \iff \frac{\sqrt{1+y_n}}{y_n} > 1 \iff \frac{1+y_n}{y_n^2} -1 > 0 \iff \\
    \iff -\frac{y_n^2-y_n-1}{y_n^2} > 0 \iff \frac{(\Phi-y_n)(y_n+\frac{\sqrt{5}}{2}-1)}{y_n^2} > 0. 
\end{multline*}

This expression is true if and only if $y_n < \Phi$.
Furthermore, we can prove that $y_n < \Phi \Rightarrow y_{n+1} < \Phi$ (the same argument as in part (c) of the problem).
We also note that $y_1 = 1 < \Phi$.
We conclude that $y_n$ is increasing.
Furthermore, $y_n$ is bounded, therefore it must converge and, furthermore, it must be a Cauchy sequence.
Therefore, denoting $\lim y_n = L$:
\begin{gather*}
    \lim_{n \to \infty} \frac{\sqrt{1+y_n}}{y_n} = 1 \iff \frac{\sqrt{1+L}}{L} = 1 \\
    L^2 - L - 1 = 0    
\end{gather*}

From this we find that $L = \dfrac{1 \pm \sqrt{5}}{2}$. Since all $y_n$ are positive:
\[ \lim y_n = \dfrac{1+\sqrt{5}}{2} = \Phi. \]

\begin{tcolorbox}
d) The Fibonacci sequence is defined by $z_1 = 1, z_2 = 1, z_{n+2} = z_{n+1} + z_n$.
Show that the sequence of ratios of successive elements, $\frac{z_{n+1}}{z_n}$, converges to $\Phi$.
\end{tcolorbox}

\begin{proof}

Consider sequence $x_n = \frac{z_{n+1}}{z_n}$.
We can rewrite this as
\[ x_n = \frac{z_{n+1}}{z_n} = \frac{z_{n}+z_{n-1}}{z_n} = 1 + \frac{z_{n-1}}{z_n} = 1 + \frac{1}{\frac{z_n}{z_{n-1}}} = 1+\frac{1}{x_{n-1}} \]
We also notice that $x_1 = 1$.
Thus, by 6b), sequence $x_n$ must converge to $\Phi$.

\end{proof}


\end{document}
