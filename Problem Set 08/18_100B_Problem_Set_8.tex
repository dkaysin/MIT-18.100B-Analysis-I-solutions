\documentclass{article}
\usepackage[utf8]{inputenc}
\usepackage[english]{babel}
\usepackage[]{amsthm}
\usepackage[]{amssymb} 
\usepackage{amsmath}
\usepackage{tcolorbox}
\usepackage{mathtools}

\newcommand{\R}{\mathbb{R}}
\DeclarePairedDelimiter{\norm}{\lVert}{\rVert}

\newtheorem{claim}{Claim}


\title{18.100B: Problem Set 8}
\author{Dmitry Kaysin}
\date{January 2020}

\begin{document}
\maketitle 

\subsection*{Problem 1}

\begin{tcolorbox}
Let $f : [0, \infty) \to \R$ be continuous, and suppose
\[ f(x^2) = f(x) \]
holds for every $x \geq 0$.
Prove that $f$ has to be a constant function.
\end{tcolorbox}

\begin{proof}

Consider sequence $a^x_n = \{ x, x^2, x^4, \dots \}$ for $x \in [0, 1)$.
We notice that $a^x_n \to 0$.
Image of any element of $a^x_n$ under $f$ must be the same value $f(x)$, thus $\lim f(a^x_n) = f(x)$.
Since $f$ is continuous at $0$,  $\lim f(a^x_n)$ must be equal to $f(0)$. Therefore
\[ f(x) = f(0) \text{ for all } x \in [0, 1) \]

Consider sequence $b^x_n = \{ x, x^{\frac{1}{2}}, x^{\frac{1}{4}}, \dots \}$ for any $x \in (0, \infty) \setminus \{1\}$.
We notice that $b^x_n \to 1$.
Image of any element of $b^x_n$ under $f$ must be the same value $f(x)$, thus $\lim f(b^x_n) = f(x)$.
Since $f$ is continuous at $1$,  $\lim f(b^x_n)$ must be equal to $f(1)$. 
Therefore
\[ f(x) = f(1) \text{ for all } x \in (0, \infty) \setminus \{1\} \]

We conclude that for any $x \in [0, \infty) : f(x) = f(0) = f(1)$, thus $f$ must be constant.

\end{proof}


\subsection*{Problem 2}

\begin{tcolorbox}
Suppose $f$ is defined and differentiable for every $x > 0$, and $f'(x) \to 0$ as $x \to + \infty$.
Put $g(x) = f(x+1) - f(x)$.
Prove that $g(x) \to 0$ as $x \to +\infty$.
\end{tcolorbox}

\begin{proof}

Consider arbitrary $\epsilon>0$.
Since $f'(x) \to 0$ as $x\to +\infty$, there exists $a$ such that for all $x>a : | f'(x)| < \epsilon$.

Now consider arbitrary point $b > a$.
We note that $f(x)$ is continuous and differentiable on $(b,b+1)$, thus applying Mean Value Theorem (Rudin 5.10):
\[ g(b) = f(b+1) - f(b) = (b+1-b) f'(z) = f'(z) \]
for some $z \in (b,b+1)$.
Since $z > b > a$, $f'(z) < \epsilon$, therefore
\[ | g(b) | < \epsilon \]
We conclude that there exists $a$ such that for all points $b>a$ absolute value of function $g$ at $b$ is less than $\epsilon$, which is exactly the definition of $g(x) \to 0$ as $x \to +\infty$.

\end{proof}


\subsection*{Problem 3}

\begin{tcolorbox}
If
\[ C_0 + \frac{C_1}{2} + \cdots + \frac{C_{n-1}}{n} + \frac{C_n}{n+1} = 0 \]
where $C_0, \dots , C_n$ are real constants, prove that the equation
\[ C_0 + C_1 x + \cdots + C_{n-1} x^{n-1} + C_n x^n = 0 \]
has at least one real root between $0$ and $1$.
\end{tcolorbox}

\begin{proof}

Consider polynomial $C(x)$ such that
\[ C'(x) = C_0 + C_1 x + \cdots + C_{n-1} x^{n-1} + C_n x^n, \]
specifically:
\[ C(x) = C_0 x + C_1 \frac{x^2}{2} + \cdots + C_{n-1} \frac{x^n}{n} + C_n \frac{x^{n+1}}{n+1} \]
Evaluate $C(x)$ at $x=0$ and $x=1$:
\begin{gather*}
    C(0) = 0 \\
    C(1) = C_0 + \frac{C_1}{2} + \cdots + \frac{C_{n-1}}{n} + \frac{C_n}{n+1} = 0
\end{gather*}
By Mean Value Theorem
\[ 0 = C(1) - C(0) = C'(z) \]
for some $z \in (0,1)$, which is equivalent to the original equation having a root between $0$ and $1$.

\end{proof}


\subsection*{Problem 4}

\begin{tcolorbox}
Suppose $f$ is a real function defined on $\R$.
We call $x \in \R$ a fixed point of $f$ if $f(x) = x$.

a) If $f$ is differentiable and $f'(t) \neq 1$ for every real $t$, prove that $f$ has at most one fixed point.
\end{tcolorbox}

\begin{proof}

Suppose there exist two fixed points of $f$, namely $x_1$ and $x_2$ and without loss of generality $x_2 > x_1$.
Applying Mean Value Theorem:
\[ f(x_2) - f(x_1) = (x_2 - x_1) f'(z) \]
for some $z \in (x_1, x_2)$. Using definition of fixed point:
\begin{gather*}
    (x_2 - x_1) = (x_2 - x_1) f'(z) \\
    1 = f'(z)
\end{gather*}
for some $z$, which contradicts the original assumption.

\end{proof}

\begin{tcolorbox}
b) Show that the function $f$ defined by
\[ f(t) = t+(1+e^t)^{-1} \]
has no fixed point, although $0 < f'(t) < 1$ for all real $t$.
\end{tcolorbox}

\begin{proof}

We notice that $f(t) > t$ for any $t \in \R$:
\[  t+(1+e^t)^{-1} - t = (1+e^t)^{-1} > 0 \]
for any $t \in \R$.
Therefore, no point of $\R$ can be a fixed point of $f$.

\end{proof}

\begin{tcolorbox}
However, if there is a constant $A < 1$ such that $|f'(t)| \leq A$ for all real $t$, prove that a fixed point $x$ of $f$ exists, and that $x = \lim x_n$, where $x_1$ is an arbitrary real number and
\[ x_{n+1} = f(x_n) \]
for $n = 1,2,3,\dots$
\end{tcolorbox}

\begin{proof}

First we notice from MVT that
\[ f(b)-f(a) = (b-a) f'(z) \]
for some $z \in (a,b)$. Since $|f'(z)| \leq A < 1$, $f$ is a contraction mapping, i.e. $|f(b)-f(a)| \leq A |b-a|$.

Next we consider difference between any two successive elements of sequence $x_n$:
\[
    d_n = |x_{n+1} - x_{n}| 
    = |f(x_{n}) - f(x_{n-1})| \leq A |x_{n}-x_{n-1}| 
    < |x_{n}-x_{n-1}|
\]
Moreover:
\[ d_n \leq A^{n-1} |x_2 - x_1| \]
Since $0<A<1$ and $|x_2-x_1|$ is finite, for any $\epsilon>0$ we can find $n$ such that $d_n < \epsilon$.
Therefore, $d_n \to 0$ and $x_n$ is Cauchy sequence.
Since $\R$ is complete, $x_n$ must converge, denote $p = \lim x_n$.
Then we notice:
\[ p = \lim x_n = \lim f(x_{n-1}) = f (\lim x_{n-1}) = f(p), \]
where the third equality holds because $f$ is differentiable and, thus, continuous.
We conclude that $p$ is a fixed point of $f$.

\end{proof}


\subsection*{Problem 5}

\begin{tcolorbox}
Let $f$ be a continuous real function on $\R$, of which it is known that $f'(x)$ exists for all $x \neq 0$
and that $f'(x) \to 3$ as $x \to 0$. Does it follow that $f'(0)$ exists?
\end{tcolorbox}

\begin{proof}

By definition:
\[ f'(0) = \lim_{x \to 0} \frac{f(x)-f(0)}{x} \]
Numerator is differentiable in the punctured neighbourhood of $0$.
Denominator is differentiable and is not equal to $0$ in the punctured neighbourhood of $0$.
Applying L'Hospital's rule:
\[ \lim_{x \to 0} \frac{f(x)-f(0)}{x} = \lim_{x \to 0} \frac{f'(x)-0}{1} =  \lim_{x \to 0} f'(x) = 3 \]
Thus, $f'(0)$ exists and is equal to $3$.

\end{proof}


\subsection*{Problem 6}

\begin{tcolorbox}
Let $f$ be a real function on $[a, b]$ and suppose $n \geq 2$ is an integer, $f^{(n-1)}$ is continuous on $[a, b]$, and $f^{(n)}(x)$ exists for all $x \in (a, b)$.
Moreover, assume there exists $x_0 \in (a, b)$ such that
\[ f'(x_0) = f''(x_0) = \cdots = f^{(n-1)}(x_0) = 0, \>\> f^{(n)}(x_0) = A \neq 0 \]
Prove the following criteria:
If $n$ is even, then $f$ has a local minimum at $x_0$ when $A > 0$, and $f$ has a local maximum at $x_0$ when $A < 0$.
If $n$ is odd, then $f$ does not have a local minimum or maximum at $x_0$.
\end{tcolorbox}

\begin{proof}

By Taylor's theorem:
\begin{multline*}
    f(x) = f(x_0) + f'(x_0)(x-x_0) + \frac{f''(x_0)}{2}(x-x_0)^2 + \cdots \\
    \cdots + \frac{f^{(n-1)}(x_0)}{(n-1)!}(x-x_0)^{n-1} + \frac{f^{(n)}(z)}{n!}(x-x_0)^n \\
    = f(x_0) + \frac{f^{(n)}(z)}{n!}(x-x_0)^n    
\end{multline*}
for some $z$ between $x_0$ and $x$.
If $n$ is even, the sign of the second term of the above expression is determined solely by the sign of $f^{(n)}(z)$.
Since $f^{(n)}$ is differentiable at $x_0$, it must be also continuous at $x_0$.
If $f^{(n)}(x_0) = A > 0$, there exists some neighbourhood of $x_0$ where all $f^{(n)}(x)>0$.
For any $x$ in such neighbourhood, $f(x) > f(x_0)$; therefore $x_0$ is a local minimum.
Conversely, if $f^{(n)}(x_0) = A < 0$, there exists some neighbourhood of $x_0$ where all $f^{(n)}(x)<0$.
For any $x$ in such neighbourhood, $f(x-x_0) < f(x_0)$; therefore $x_0$ is a local maximum.

However, if $n$ is odd, expression $\frac{f^{(n)}(z)}{n!}(x-x_0)^n$ changes sign when going from $x < x_0$ to $x > x_0$.
Therefore, in any neighbourhood of $x_0$ there are points $x$ such that $f(x)>f(x_0)$ and there are points $x$ such that $f(x)<f(x_0)$.
Therefore, $x_0$ can be neither local minimum nor local maximum.

\end{proof}


\subsection*{Problem 7}

\begin{tcolorbox}
For $f(x) = |x|^3$, compute $f'(x), f''(x)$ for all real $x$, and show that $f^{(3)}(0)$ does not exist.
\end{tcolorbox}

\begin{align*}
    f(x) =
    |x|^3 = |x^3| = & \begin{cases}
        x^3, & \text{ if $x \geq 0$} \\
        -x^3, & \text{ if $x < 0$}
    \end{cases} \\
    f'(x) = &
    \begin{cases}
        3x^2, & \text{ if $x \geq 0$} \\
        -3x^2, & \text{ if $x < 0$}
    \end{cases} \\
    f''(x) = &
    \begin{cases}
        6x, & \text{ if $x \geq 0$} \\
        -6x, & \text{ if $x < 0$}
    \end{cases} \\
    f^{(3)}(x) = & 
    \begin{cases}
        6, & \text{ if $x > 0$} \\
        -6 & \text{ if $x < 0$}
    \end{cases}
\end{align*}
We notice that $f^{(3)}(0)$ does not exist since right-hand and left-hand derivatives of $f''(x)$ differ at $x=0$:
\begin{align*}
    f_{x \to 0^+}^{(3)}(x) & = \lim_{x \to 0^+} \frac{f''(x)-f''(0)}{x-0} = \lim_{x \to 0^+} \frac{6x}{x} = 6 \\
    f_{x \to 0^-}^{(3)}(x) & = \lim_{x \to 0^-} \frac{f''(x)-f''(0)}{x-0} = \lim_{x \to 0^-} \frac{-6x}{x} = -6
\end{align*}


\end{document}
